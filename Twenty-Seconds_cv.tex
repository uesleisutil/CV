% !TEX root = /media/ueslei/Ueslei/CV/CV_EN/Twenty-Seconds_cv.tex
\documentclass[letterpaper]{twentysecondcv} % a4paper for A4
\usepackage{microtype} % Slightly tweak font spacing for aesthetics
\usepackage[utf8]{inputenc} % Required for including letters with accents
\usepackage[T1]{fontenc} % Use 8-bit encoding that has 256 glyphs
\usepackage{gensymb}
\hypersetup{
    colorlinks,
    linkcolor={red!50!black},
    citecolor={blue!50!black},
    urlcolor={blue!80!black}
}
%----------------'--------------------------------------------------------------------
%	 PERSONAL INFORMATION
%-------------------------------------------------------------------------------------

\profilepic{ueslei_2.jpg}
\cvname{Ueslei Sutil}
\cvjobtitle{Oceanographer}
\cvdate{1990 August 22}
\cvaddress{São José dos Campos, SP, Brazil}
\cvnumberphone{+55 12 98199 2431}
\cvsite{\textcolor{mainblue}{\href{https://uesleisutil.github.io/}{uesleisutil.github.io}}}
\cvmail{ueslei.sutil@inpe.br}

%----------------------------------------------------------------------------------------

\begin{document}

%------------------------------------------------------------s----------------------------
%	 ABOUT ME
%----------------------------------------------------------------------------------------
\aboutme{
• Air-sea-ice interactions;\\
• Oceanographic and meteorological instrumentation;\\
• Numerical modeling (e.g. COAWST, WRF, ROMS, SWAN and WW3).
}

%----------------------------------------------------------------------------------------
%	 SKILLS
%----------------------------------------------------------------------------------------
\skills{{Tensorflow/3.5},{Fortran/5},{MATLAB/5.1},{Shell Script/5.5},{\LaTeX/6},{NetCDF Operators (NCO)/6}, {Python/6},{NCAR Command Language (NCL)/6},  {COAWST (WRF, ROMS, SWAN, WW3)/6}, {Italian language/5.2},{\textcolor{mainblue}{\href{https://www.efset.org/cert/PrWqbW}{English language (C2 Level)}} /5.5},{Portuguese language (Native)/6}}
\makeprofile % Print the sid

%---------------------------------------------------------------------------------------
%	 INTERESTS
%----------------------------------------------------------------------------------------
\section{About me\textcolor{maingray}{................................................................................}}

I'm a physical oceanographer with programming language and data analysis skills. The main object of my work is ocean-atmosphere interaction during extreme events in the South Atlantic and Antarctica using coupled regional numerical modeling.
\\

%----------------------------------------------------------------------------------------
%	 EDUCATION
%----------------------------------------------------------------------------------------
\section{Education\textcolor{mainblue}{...............................................................................}}

\begin{twenty} % Environment for a list with descript

\twentyitem{2014-2016}{Master's degree}{Federal University of Rio Grande do Sul}{Remote sensing with emphasis in Meteorology.\\ Thesis: \textit{Ocean-Atmosphere Interaction in an Extratropical Cyclone in the Southwest Atlantic: a  high resolution numerical modeling approach}. \\
   Advisors: Dr. Luciano Ponzi Pezzi and Dra. Rita de Cássia Marques Alves.\\You can access a copy \textcolor{mainblue}{\href{http://www.bibliotecadigital.ufrgs.br/da.php?nrb=001049476&loc=2017&l=402d86e18a39851e}{here}}.}\\

\twentyitem{2009-2013}{Bachelor's degree}{Federal University of Paraná}{Oceanography \\ Thesis: \textit{Sea Surface Temperature variability during an extreme precipitation event in Santa Catarina, Brazil}. \\ Advisors: Dr. Marcelo Sandin Dourado and Dr. Dirceu Luis Severo.\\ You can access a copy \textcolor{mainblue}{\href{https://www.researchgate.net/publication/329070542_VARIABILIDADE_DA_TEMPERATURA_DA_SUPERFICIE_DO_MAR_DURANTE_UM_EVENTO_EXTREMO_DE_PRECIPITACAO_EM_SANTA_CATARINA}{here}}.}
\end{twenty}

%---------------------------------------------------------------------------------------
%	 EXPERIENCE
%----------------------------------------------------------------------------------------

\section{Professional experience\textcolor{maingray}{........................................................}}

\begin{twenty} % Environment for a list with desc

\twentyitem{2017-Today}{Assistant research fellow}{National Institute for Space Research}{
• Implementation of the WaveWatch 3 and ROMS Sea Ice models; \\ • Hydrodynamic and atmospheric modeling using ROMS, WRF and WaveWatch to study air-sea-ice interactions in the Weddell Sea.}\\

\twentyitem{2016-2017}{Assistant research fellow}{National Institute for Space Research}{
• Implementation of the COAWST Modeling System in a parallel cluster in order to bring the state-of-the-art in numerical modeling to the Institute; \\ • Hydrodynamic and atmospheric modeling using ROMS, SWAN and WRF to study extreme weather events and their air-sea interaction in the Southwest Atlantic.}\\

\twentyitem{2014-2016}{Graduate research fellow}{Federal University of Rio Grande do Sul}{
 Data processing and analysis of extreme precipitation events in South Brazil.}\\

\twentyitem{2012-2013}{Undergraduate research fellow}{Federal University of Paraná}{
 Monitor of the Meteorology class of the Oceanography graduation course for 2012 and 2013 classes.}\\
\end{twenty}

%----------------------------------------------------------------------------------------
%	 Main publications
%---------------------------------------------------------------------------------------
\section{Publications\textcolor{mainblue}{............................................................................}}

Sutil, U. A.\textbf{ and L. P. Pezzi. 2018. \textit{Practical guide for COAWST configuration - 2ª edition (in Portuguese)}. São José dos Campos: INPE, 2019, 100p., ISBN: <978-85-17-00098-0> doi: \textcolor{mainblue}{\href{http://dx.doi.org/10.13140/RG.2.2.15147.18720}{10.13140/RG.2.2.15147.18720}}}.

Sutil, U. A.\textbf{, L. P. Pezzi, R. C. M. Marques and A. B. Nunes. 2019. Ocean-atmosphere interactions in an extratropical cyclone in the Atlantic Southwest. \textit{Anuário de Geociências da UFRJ, 42}(1): 525-535, doi: \textcolor{mainblue}{\href{http://dx.doi.org/10.11137/2019_1_525_535}{10.11137/2019\_1\_525\_535}}}.


Sutil, U. A.\textbf{ and L. P. Pezzi. 2018. \textit{Practical guide for COAWST configuration (in Portuguese)}. São José dos Campos: INPE, 2018, 91p, doi: \textcolor{mainblue}{\href{https://doi.org/10.13140/rg.2.2.31726.87363}{10.13140/rg.2.2.31726.87363}}}.


%----------------------------------------------------------------------------------------
%	 Seminars
%----------------------------------------------------------------------------------------
\newpage
\makeprofile
\section{Conference and seminar presentations\textcolor{maingray}{...............................}}

\textbf{Cabrera, M. J., L. P. Pezzi, M. F. Santini and} U. A. Sutil\textbf{. The role of the air temperature advection and the Sea Surface Temperature local modulation on the Marine Atmospheric Boundary Layer at the Atlantic Southwest. XI Brazilian Micrometeorology Workshop, São José dos Campos-Brazil, 2018.} \\

Sutil, U. A.\textbf{ and L. P. Pezzi. The Brazil Current's warming role in amplifying the November 2008 precipitation event in Santa Catarina. 10th International Workshop on Modeling the Ocean, Santos-Brazil, doi: \textcolor{mainblue}{\href{https://www.researchgate.net/publication/329070547_Role_of_Brazil_Current_warming_in_amplifying_2008_Santa_Catarina_extreme_precipitation_event}{10.13140/RG.2.2.21895.24483}}, 2018.}

Sutil, U. A.\textbf{, L. P. Pezzi and R. C. M. Alves. Sea Surface Temperature impact during a case of cyclogenesis in the Southwest Atlantic through numerical modeling in high and low resolution. III Meeting of the National Institute of Cryosphere Science and Technology, Cambará do Sul-Brazil, 2015.}

Sutil, U. A.\textbf{, L. P. Pezzi and R. C. M. Alves. Southwest Atlantic Sea Surface Temperature variability through numerical modeling. 11th International Conference on Southern Hemisphere Meteorology and Oceanography, Santiago-Chile, 2015.}

Sutil, U. A.\textbf{, M. S. Dourado, D. L. Severo and R. C. M. Alves. Sea Surface Temperature variability during a extreme precipitation event in Santa Catarina, Brazil. VI Brazilian Oceanography Congress, Itajaí-Brazil, 2014.}

Sutil, U. A.\textbf{ and M. S. Dourado. Meteorological aspects associated to the November 2008 extreme precipitation event in Santa Catarina. 5° Integrated Week of Teaching, Research and Extension, Curitiba-Brazil, doi: \textcolor{mainblue}{\href{https://www.researchgate.net/publication/329070463_Aspectos_meteorologicos_associados_ao_evento_extremo_de_novembro_de_2008_no_leste_do_Estado_de_Santa_Catarina}{10.13140/RG.2.2.25250.68800}}, 2013.}
\\
%----------------------------------------------------------------------------------------
%	 Posters
%----------------------------------------------------------------------------------------
%\newpage
%\makeprofile
\\\\
\section{Poster presentations\textcolor{mainblue}{............................................................}}


\textbf{Cabrera, M.J., L. P. Pezzi, M. F. Santini and} U. A. Sutil\textbf{. Physical mechanisms responsible for the stability of the Marine Atmospheric Boundary Layer in Southwest Atlantic region. XVI Meeting of postgraduate students in Meteorology at CPTEC/INPE, Cachoeira Paulista-Brazil, 2017.}\

Sutil, U. A.\textbf{, L. P. Pezzi, R. C. M. Alves. Ocean-atmosphere interactions in an extratropical cyclone in the Southwest Atlantic: a high resolution numerical approach. XVII Brazilian Symposium in Remote Sensing. Proceedings of the XVII Brazilian Symposium in Remote Sensing, p. 5666-5673, ISBN 978-85-15-00088-1. Avaliable at:<\textcolor{mainblue}{\href{http://urlib.net/rep/8JMKD3MGP6W34M/3PSMBG8?ibiurl.language=pt-BR}{http://urlib.net/rep/8JMKD3MGP6W34M/3PSMBG8?ibiurl.language=pt-BR}}>.}

Sutil, U. A .\textbf{, L. P. Pezzi, R. C. M. Alves, M. Fagundes and M. B. Gouveia. Variability of the Sea Surface Temperature during a cyclogenic event through high resolution numerical modeling. XVI Meeting of postgraduate students in Meteorology at CPTEC/INPE, Cachoeira Paulista-Brazil, 2015.}

\textbf{Fagundes, M., P. C. Campos, C. K. Parise, L. P. Pezzi, A. R. T. Junior,} U. A. Sutil\textbf{ and M. B. Gouveia. Oceanic surface circulation of the Equatorial Atlantic in extreme periods of El Niño and La Niña: Preliminary results. XI Symposium of Waves, Tides, Ocean Engineering and Satellite Oceanography, Arraial do Cabo-Brazil, 2015.}

\textbf{Gouveia, M. B., M. Fagundes, P. C. Campos and} U. A. Sutil\textbf{. Ocean boundary forcing analysis. XI Symposium of Waves, Tides, Ocean Engineering and Satellite Oceanography, Arraial do Cabo-Brazil, 2015.}

Sutil, U. A.\textbf{, M. S. Dourado and D. L. Severo. Climatological study of Sea Surface Temperature and Latent Heat Fluxes in the South and Southeast regions of Brazil. XVII Brazilian Congress of Meteorology, Recife-Brazil, 2014.}

Sutil, U. A .\textbf{, L. P. Pezzi and R. C. M. Alves. Variability of the Sea Surface Temperature through numerical modeling: preliminary results. XII Argentine Congress of Meteorology, Mar del Plata-Argentina, 2014.}

Sutil, U. A.\textbf{, M. S. Dourado and  D. L. Severo. Atlantic Southwest Sea Surface Temperature climatology. X Symposium of Waves, Tides, Ocean Engineering and Satellite Oceanography, Arraial do Cabo-Brazil, 2013.}

\newpage
\makeprofile

Sutil, U. A .\textbf{, M.S. Dourado and D. L. Severo. Climatology of the Sea Surface Temperature in South Brazil. V International Symposium of Climatology, Florianópolis-Brazil, 2013.}

Sutil, U. A.\textbf{, M. S. Dourado and D. L. Severo. Climatology of the Sea Surface Temperature in the states of Paraná and Santa Catarina. Brazilian Congress of Oceanography,  Rio de Janeiro-Brazil, 2012.}

\textbf{Cazal, H. S. V., B. A. Oliveira, I. M. Fomin,} U. A. Sutil\textbf{, E. Oliveira and M. S. Dourado. Seasonal Characterization of Precipitation and Wind in the Parana Coast. XIV Latin American Congress of Marine Sciences, Baneário Camboriú-Brazil, 2011.}\\

%-------------------------------------------------------------------------------------
% Participation in Course Completion Works Examination Boards
%----------------------------------------------------------------------------------------
\section{Course completion works examination boards\textcolor{maingray}{...................}}


\begin{twentyshort}
	\twentyitem{2017}{Course conclusion paper}{Federal University of Paraná}{Sutil, U. A.\textbf{, C. R. Soares and M. V. Silva. Participation in the examination board of Andressa Bernd's course conclusion paper: Study of the fog formation at the Paranaguá Estuary Complex. Graduation in Oceanography.}}\\
\end{twentyshort}
\\
%-------------------------------------------------------------------------------------
%	 Experience at the sea
%----------------------------------------------------------------------------------------
%\newpage
%\makeprofile
\section{Cruises\textcolor{mainblue}{..................................................................................}}

\begin{twentyshort}
	\twentyitem{2019}{38th Antarctic Operation}{Polar Research Vessel Almirante Maximiano (H41)}{• Researcher in the ATMOS/INPE project (Antartic Modeling and Observation System), collecting \textit{in-situ} data at Antartica (BMC) with radiosonde, oceanographic buoy, a high frequency meteorological tower and a low cost Arduino device; \\ • Duration: 55 days and 2510 miles covered.}\\
    \twentyitem{2018}{37th Antarctic Operation}{Polar Research Vessel Almirante Maximiano (H41)}{• Researcher in the INTERCONF/INPE project (Ocean-Atmosphere Interaction at the Brazil-Malvinas Confluence), collecting \textit{in-situ} data in the Brazil-Malvinas Confluence (BMC) with XBT, CTD, radiosonde and a high frequency meteorological tower; \\ • Data processing and analysis with Python and NCL; \\ • Duration: 24 days and 3580 miles covered.}\\
	\twentyitem{2018}{36th Antarctic Operation}{Polar Research Vessel Almirante Maximiano (H41)}{• Researcher in the INTERCONF project, collecting in-situ data in the BMC and Antarctica with XBT, CTD, radiosonde and a high frequency meteorological tower; \\ • Data processing and analysis with NCL and MATLAB; \\ • Duration: 24 days and 4028 miles covered.}\\
	\twentyitem{2017}{35th Antarctic Operation}{Polar Research Vessel Almirante Maximiano (H41)}{• Researcher in the INTERCONF project, collecting \textit{in-situ} data in the BMC and Antarctica with XBT, radiosonde; \\ • Data processing and analysis with NCL;\\ • Duration: 14 days and 2736 miles covered.}\\
	\twentyitem{2016}{34th Antarctic Operation}{Polar Research Vessel Almirante Maximiano (H41)}{• Researcher in the INTERCONF project, collecting \textit{in-situ} data in the BMC with XBT, radiosonde and a meteorological tower;\\ • Duration: 16 days and 2567 miles covered.}\\
\end{twentyshort}
\\
%----------------------------------------------------------------------------------------
%	 COURSES
%----------------------------------------------------------------------------------------
\newpage
\makeprofile

\section{Courses and training\textcolor{maingray}{.............................................................}}

\begin{twentyshort}
	\twentyitemshort{2019}{\textcolor{black}{Python language: 2.7 and 3.7.}}\\
	\twentyitemshort{2017}{\textcolor{black}{Introduction to Python language.}}\\
	\twentyitemshort{2016}{\textcolor{black}{NCAR Command Language (NCL).}}\\
	\twentyitemshort{2014}{\textcolor{black}{Investigating oceanic properties through remote sensing.}}\\
	\twentyitemshort{2013}{\textcolor{black}{Antarctica's influence on climate: past, present and future.}}\\
\end{twentyshort}

\begin{twentyshort}
	\twentyitemshort{2013}{\textcolor{black}{Processing satellite images using MATLAB.}}\\
	\twentyitemshort{2012}{\textcolor{black}{Interpretation of synoptic charts.}}\\

	\twentyitemshort{2012}{\textcolor{black}{Dynamics of the Antarctic Circumpolar Current.}}\\
	\twentyitemshort{2012}{\textcolor{black}{Satellite oceanography.}}\\
	\twentyitemshort{2012}{\textcolor{black}{Earth magnetism.}}\\
	\twentyitemshort{2011}{\textcolor{black}{Introduction to the numerical modeling.}}\\
\end{twentyshort}

%----------------------------------------------------------------------------------------
%	 VOLUNTEER EXPERIENCE
%----------------------------------------------------------------------------------------
%\section{Referees\textcolor{mainblue}{.................................................................................}}

%\begin{twentyshort}

%	\twentyitem{1\textsuperscript{st}}{Dr. Marcelo Sandin Dourado}{Federal University of Paraná}{Email: \href{mailto:dourado@ufpr.br}{dourado@ufpr.br}}\\
%	\twentyitem{2\textsuperscript{nd}}{Dr. Luciano Ponzi Pezzi}{National Institute for Space Research}{Email: \href{mailto:luciano.pezzi@inpe.br}{luciano.pezzi@inpe.br}}\\
%	\twentyitem{3\textsuperscript{rd}}{Dr. Marcelo Santini}{National Institute for Space Research}{Email: \href{mailto:santini.marcelo@gmail.com}{santini.marcelo@gmail.com}}\\
%\end{twentyshort}


%----------------------------------------------------------------------------------------
%	 VOLUNTEER EXPERIENCE
%-----------------------------------------------------------------------------------
\section{Professional membership\textcolor{mainblue}{......................................................}}

\begin{twentyshort}

	\twentyitem{2018-Today}{\textcolor{black}{Brazilian Association of Oceanography}}{}{}\\
\end{twentyshort}
%----------------------------------------------------------------------------------------
%	 VOLUNTEER EXPERIENCE
%----------------------------------------------------------------------------------------
\section{Volunteer experience\textcolor{maingray}{............................................................}}

\begin{twentyshort}
\twentyitem{2019-2020}{Employability and strategic partnerships}{Casa 1}{Casa 1 is a Brazilian non-profit center in São Paulo that offers shelter for LGBTQ+ people that have been expelled from their homes. In our working group, we seek partnerships with individuals and companies in order to help them find job opportunities and rebuild their lifes independently. Moreover, we have worked to find new ways to fund our operations through crowdfunding, events and strategic partnerships.\\ Our site: \textcolor{mainblue}{\href{http://www.casaum.org/}{http://www.casaum.org/}}}\\
\end{twentyshort}
\null\vfill
\begin{center}
	{Ueslei Adriano Sutil~~~·~~~CV \\ Updated: April 20 2020}
\end{center}
\end{document}
\end{document}
